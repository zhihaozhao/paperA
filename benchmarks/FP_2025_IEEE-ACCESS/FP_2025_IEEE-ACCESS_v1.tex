%% IEEE Access Style Template (using standard article class)
%% Compatible with pdfLaTeX and MikTeX

\documentclass[10pt,twocolumn]{article}

% Standard packages for IEEE-style formatting
\usepackage{cite}
\usepackage{amsmath,amssymb,amsfonts}
\usepackage{algorithm}
\usepackage{algpseudocode}
\usepackage{graphicx}
\usepackage{float}
\usepackage{array}
\usepackage{url}
\usepackage[margin=0.75in]{geometry}

% Algorithm formatting
\renewcommand{\algorithmicrequire}{\textbf{Input:}}
\renewcommand{\algorithmicensure}{\textbf{Output:}}

% IEEE-style title formatting
\makeatletter
\def\@maketitle{%
  \newpage
  \null
  \vskip 2em%
  \begin{center}%
  \let \footnote \thanks
    {\Large \@title \par}%
    \vskip 1.5em%
    {\large
      \lineskip .5em%
      \begin{tabular}[t]{c}%
        \@author
      \end{tabular}\par}%
    \vskip 1em%
  \end{center}%
  \par
  \vskip 1.5em}
\makeatother

\title{Perception-to-Action Benchmarks for Autonomous Fruit-Picking Robots: Quantitative Synthesis, Gaps, and Deployment Roadmap}

\author{
  Zhihao Zhao$^{1,2}$, Yanxiang Zhao$^{3}$, and Nur Syazreen Ahmad$^{1}$ \\
  \\
  $^{1}$School of Electrical and Electronic Engineering, \\
  Universiti Sains Malaysia, 14300 Nibong Tebal, Penang, Malaysia \\
  $^{2}$YanTai Engineering and Technology College, 264006 YanTai, Shandong, China \\
  $^{3}$Central South University, Changsha, Hunan, 410083, China \\
  \\
  \textit{Corresponding author: Nur Syazreen Ahmad (syazreen@usm.my)}
}

\begin{document}

\maketitle

\begin{abstract}
This review synthesizes recent advances in autonomous fruit-picking robots with a focus on visual perception, path planning, and motion control. Following PRISMA guidelines, we systematically analyzed 137 studies published between 2015 and 2024. We critically compare learning-based perception approaches---especially YOLO- and R-CNN family methods---under orchard-specific conditions (occlusion, variable lighting, small targets), and quantify trade-offs in accuracy and efficiency across representative platforms. We further examine perception-to-action integration for motion planning and control, summarizing success rate, harvest cycle time, and damage rate reported in field and greenhouse trials. 

Learning-based approaches, including transfer learning and reinforcement learning (e.g., Deep Deterministic Policy Gradient), have enhanced the generalizability of robotic arm motion planning for collision-free harvesting. Innovative path-planning algorithms and robust control strategies enable autonomous robots to navigate unstructured environments and compensate for real-time disturbances, increasing system reliability. Despite these advances, persistent challenges remain in multi-source data integration, delicate fruit handling, and cost-effective scalability. This survey provides a comprehensive evaluation of technological progress, identifies critical research gaps in deployment and scalability, and proposes future directions to accelerate commercial adoption in diverse agricultural contexts.
\end{abstract}

\textbf{Keywords:} Agricultural robotics, autonomous fruit picking, computer vision, deep learning, motion planning, perception-action integration, R-CNN, YOLO, robotic harvesting, machine learning

% ... existing code ...