% Clean IEEE Access Preamble - No Conflicts
\documentclass[conference]{IEEEtran}

% Essential packages in proper order
\usepackage{cite}
\usepackage{amsmath,amssymb,amsfonts}
\usepackage{graphicx}
\usepackage{textcomp}
\usepackage{xcolor}
\usepackage{url}

% Table packages (before natbib to avoid conflicts)
\usepackage{multirow}
\usepackage{array}
\usepackage{booktabs}
\usepackage{threeparttable}
\usepackage{rotating}
\usepackage{longtable}
\usepackage{tabularx}

% Bibliography (after table packages)
\usepackage[numbers,sort&compress]{natbib}

% Algorithm packages (after natbib)
\usepackage{algorithm}
\usepackage{algpseudocode}
\renewcommand{\algorithmicrequire}{\textbf{Input:}}
\renewcommand{\algorithmicensure}{\textbf{Output:}}

% Float package last
\usepackage{float}

% NO cleveref (conflicts with natbib in IEEEtran)
% NO duplicate array package
% NO conflicting packages

\sloppy

\begin{document}

\title{Perception-to-Action Benchmarks for Autonomous Fruit-Picking Robots: Quantitative Synthesis, Gaps, and Deployment Roadmap}

\author{\IEEEauthorblockN{Zhihao Zhao}
\IEEEauthorblockA{\textit{School of Electrical and Electronic Engineering} \\
\textit{Universiti Sains Malaysia}\\
Nibong Tebal, Malaysia \\
zzhaoooooo@gmail.com}
\and
\IEEEauthorblockN{Nur Syazreen Ahmad}
\IEEEauthorblockA{\textit{School of Electrical and Electronic Engineering} \\
\textit{Universiti Sains Malaysia}\\
Nibong Tebal, Malaysia \\
syazreen@usm.my}}

\maketitle

\begin{abstract}
This review provides comprehensive quantitative benchmarking of visual perception and motion control algorithms for autonomous fruit-picking robots, analyzing 56 studies published between 2015 and 2024. We present experimental validation of algorithm family performance characteristics, including R-CNN, YOLO, hybrid approaches, and traditional methods across diverse agricultural environments. Statistical analysis reveals significant performance hierarchies with YOLO algorithms achieving optimal balance (90.9% accuracy, 84ms processing) for real-time commercial deployment, while R-CNN approaches provide superior precision (90.7% accuracy) for quality-critical applications. Motion planning analysis demonstrates success rates ranging from 58-92% across different algorithmic approaches, with DDPG and reinforcement learning showing enhanced adaptability for unstructured environments. Despite technological advances, persistent challenges remain in multi-sensor fusion, cost-effective scalability, and delicate fruit handling for commercial agricultural adoption.
\end{abstract}

\begin{IEEEkeywords}
Agricultural robotics, autonomous fruit picking, computer vision, deep learning, motion planning, perception-action integration, R-CNN, YOLO, robotic harvesting, machine learning
\end{IEEEkeywords}

% CONTENT STARTS HERE - insert sections after this line