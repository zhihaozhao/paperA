\documentclass[Afour,sageh,times]{sagej}

\usepackage{moreverb,url}

\usepackage[colorlinks,bookmarksopen,bookmarksnumbered,citecolor=red,urlcolor=red]{hyperref}

\usepackage{float}
\usepackage{multirow}
\usepackage{url}
\usepackage{verbatim}
\usepackage{graphicx}
%\usepackage{cite}
\usepackage{array, longtable, tabularx}
\usepackage{booktabs}
\usepackage{xcolor} 
\usepackage{ulem} 
\usepackage{amssymb}  
%\usepackage{unicode-math}  
\definecolor{myGreen}{RGB}{0,128,0}  % 自定义绿色,RGB 值可按需调整
\definecolor{myOrange}{RGB}{255,65,0}  % 自定义橙色,RGB 值可按需调整

\newcommand\BibTeX{{\rmfamily B\kern-.05em \textsc{i\kern-.025em b}\kern-.08em
T\kern-.1667em\lower.7ex\hbox{E}\kern-.125emX}}

\def\volumeyear{2025}

\begin{document}

\runninghead{Zhao, Ahmad and Chen}

\onecolumn
%\title{A Survey on Device-free Human Activity Recognition via Wi-Fi-based Channel State Information}
\title{Response to Reviewers}
\date{}
\maketitle

%\author{Zhihao Zhao\affilnum{1,}\affilnum{2}, Nur Syazreen Ahmad\affilnum{1} and  Yabing Chen\affilnum{2}}

%\affiliation{\affilnum{1}School of Electrical and Electronic Engineering, Universiti Sains Malaysia, 14300 Nibong Tebal, Penang, Malaysia.\\
%\affilnum{2}YanTai Engineering and Technology College, 264006 YanTai, Shandong, China.}

%\corrauth{Nur Syazreen Ahmad, School of Electrical and Electronic Engineering, Universiti Sains Malaysia, 14300 Nibong Tebal, Penang, Malaysia.}

%\email{syazreen@usm.my}





% \begin{document}
\section*{Original Manuscript ID: }  [No. AIS-250186]
\section*{Original Article Title: }  "A Survey on Device-free Human Activity Recognition via Wi-Fi-based Channel State Information"


\section*{To: }Journal of Ambient Intelligence and Smart Environments
%textbf{To:} Journal of Ambient Intelligence and Smart Environments

\section*{Re: }Response to reviewers \\
\bigskip


\noindent Dear Professor Augusto, \\
  
We are grateful for the opportunity to resubmit this manuscript and address the comments of the reviewers.

We extend our gratitude to the reviewers for their insightful comments and suggestions, which have contributed to a substantial enhancement in the quality and clarity of our manuscript. A thorough evaluation of each point raised has been conducted, leading to the implementation of pertinent revisions. In the following section, a comprehensive response is provided to each of the aforementioned comments by incorporating structural improvements, additional figures, expanded critical analysis, and focused updates. These updates address concerns about recency, balance, and readability.\\ 

We are submitting
\begin{enumerate}
\item a point-by-point response to the comments as a response to reviewers in the "Response letter.pdf";
\item an updated manuscript with changes highlighted in color as the "HighLight-AIS.pdf," and
\item a clean updated manuscript without any highlights as the "Main Manuscript."
\end{enumerate}

We hope the revisions made are acceptable and align with your expectations.\\

Thank you.\\

\noindent Best regards,\\

\noindent Zhihao Zhao\\
Nur Syazreen Ahmad\\
Yabing Chen

\clearpage
\setlength{\parindent}{0pt}
\section*{Response to Reviewer \#1}  
\bigskip
%11111111111111111
\textcolor{myGreen}{ \textbf{Reviewer\#1, Concern\#1}: In abstract, there should be a space before (DFHAR) and (CSI). }
%\textbf{Response:} Thank you for pointing out this formatting issue. In the original draft (\texttt{SAGE\_V1.1.pdf}), the abstract lacked spaces before these abbreviations, which could affect readability. We have corrected this in the revised manuscript (\texttt{Main Manuscript}) by adding spaces before (DFHAR) and (CSI) in the abstract. This ensures consistency with academic formatting standards and improves overall clarity.
\\

\textbf{Author response:} Thank you for pointing out this formatting issue. In the original draft, the abstract lacked spaces before these abbreviations, which could affect readability.\\
\textbf{Author action:} We have corrected this in the revised manuscript by adding spaces before (DFHAR) and (CSI) in the abstract. This ensures consistency with academic formatting standards and improves overall clarity. The modification can be viewed on pages 1, Abstract section.\\

\textcolor{myOrange}{ In the “HighLight-AIS.pdf”, the modifications made are highlighted and annotated with “Reviewer 1: Response 1”. }\\
  
\color{gray}\rule{\linewidth}{1pt}\normalcolor\\

%22222222222222222222
\textcolor{myGreen}{ \textbf{Reviewer\#1, Concern\#2}: In keywords, there should be a space before (CSI).}  \\
 
\textbf{Author response:} We appreciate this observation. The original keywords section omitted the space before (CSI), leading to potential formatting inconsistencies.\\
\textbf{Author action:} In the revised manuscript, we have added the space before (CSI) to align with standard abbreviation presentation and enhance readability. The modification can be viewed on pages 1, Keywords.\\

\textcolor{myOrange}{In the “HighLight-AIS.pdf”, the modification made is highlighted and annotated with “Reviewer 1: Response 2”. }\\
 
\color{gray}\rule{\linewidth}{1pt}\normalcolor\\

%33333333333333
\textcolor{myGreen}{ \textbf{Reviewer\#1, Concern\#3}: In Table 6, the entry year is incorrect, i.e., Moshiri et al. 20220.}  \\

\textbf{Author response:} This typographical error has been rectified. In the original draft, the year for Moshiri et al. was incorrectly listed as 20220, likely due to a transcription mistake.\\
\textbf{Author action:} In the revised manuscript, we have corrected it to 2020, ensuring accuracy in the references and maintaining the integrity of the literature review. This change is part of broader updates to Table 6, which now focuses on recent DL models (2020-2024) with added columns for 'Key Innovation' to highlight advancements like attention mechanisms. The modification can be viewed on pages 9, Table 5 and page 10, Table 6.\\

\textcolor{myOrange}{In the “HighLight-AIS.pdf”, the modifications made are highlighted and annotated with “Reviewer 1: Response 3”. }\\

\color{gray}\rule{\linewidth}{1pt}\normalcolor\\

%44444444444
\textcolor{myGreen}{ \textbf{Reviewer\#1, Concern\#4}: On page 6, there should be a space before (USRP).}  \\
 
\textbf{Author response:} Corrected as suggested. The original draft on page 6 (corresponding to discussions of data acquisition) lacked a space before (USRP).\\
\textbf{Author action:} We have added the space in the revised manuscript to improve formatting consistency and readability. The modification can be viewed on pages 5, paragraph 3.\\

\textcolor{myOrange}{In the “HighLight-AIS.pdf”, the modifications made are highlighted and annotated with “Reviewer 1: Response 4”. }\\

%(Old: SAGE\_V1.1.pdf page 6 paragraph 1; New: Main Manuscript page 5 paragraph 3)  
\color{gray}\rule{\linewidth}{1pt}\normalcolor\\

\textcolor{myGreen}{ \textbf{Reviewer\#1, Concern\#5}: On page 5, there should be a space before (COTS), and the first letter should be capitalized, just like other abbreviations in the work.}  \\
 
\textbf{Author response:} We have addressed this by adding a space before (COTS) and capitalizing the first letter to match other abbreviations (e.g., CSI, RSSI). In the original draft, this was inconsistent on page 5 (in the context of RSSI vs. CSI discussions).\\
\textbf{Author action:} The revised manuscript now ensures uniformity across all sections, including the new discussions on WiFi protocols and AIoT frameworks. The modification can be viewed on pages 5, paragraph 3.\\

\textcolor{myOrange}{ In the “HighLight-AIS.pdf”, the modifications made are highlighted and annotated with “Reviewer 1: Response 5”. }\\

%(Old: SAGE\_V1.1.pdf page 5 paragraph 2; New: Main Manuscript page 6 paragraph 3)  
\color{gray}\rule{\linewidth}{1pt}\normalcolor\\

\textcolor{myGreen}{ \textbf{Reviewer\#1, Concern\#6}: On page 6, you provided the full name and abbreviation of IoT. So, why not provide the full name of AIoT when it first appeared? I think IoT is more common than AIoT.}  \\

%\textbf{Response:} We agree that consistency in introducing abbreviations is important, especially since IoT was fully expanded on page 6 in the original draft. In \texttt{Main Manuscript}, we have added the full name "Artificial Intelligence of Things (AIoT)" at its first appearance on the corresponding page, acknowledging IoT's commonality while clarifying AIoT's role in emerging HAR applications. This is part of the revised Section 3, which emphasizes lightweight AIoT frameworks for efficiency.% (Perumal et al. 2022 from SAGE\_V1.1.pdf chunk\_id="7"; Arshad et al. 2022 from Main Manuscript chunk\_id="1").  
\textbf{Author response:} We agree that consistency in introducing abbreviations is important, especially since IoT was fully expanded on page 6 in the original draft.\\
\textbf{Author action:} In the revised manuscript, we have added the full name "Artificial Intelligence of Things (AIoT)" at its first appearance on the corresponding page, acknowledging IoT's commonality while clarifying AIoT's role in emerging HAR applications. The modification can be viewed on pages 1, Abstract.\\

\textcolor{myOrange}{
In the “HighLight-AIS.pdf”, the modifications made are highlighted and annotated with “Reviewer 1: Response 6”. }\\%(Old: SAGE\_V1.1.pdf page 6 paragraph 3; New: Main Manuscript page 2 paragraph 4)  

\color{gray}\rule{\linewidth}{1pt}\normalcolor\\

\textcolor{myGreen}{ \textbf{Reviewer\#1, Concern\#7}: Repeated abbreviations, such as Generative Adversarial Networks (GANs), appear twice on page 13.}  \\

%\textbf{Response:} This redundancy has been removed. In SAGE\_V1.1.pdf, GANs was repeated on page 13 (in discussions of learning-based methods).\\
%We eliminated it In the revised manuscript  to streamline the text and avoid repetition. This aligns with updates to Section 1, where GANs are now first abbreviated.\\
\textbf{Author response:} This redundancy has been removed. In the original draft, GANs was repeated on page 13 (in discussions of learning-based methods).\\
\textbf{Author action:} We eliminated it in the revised manuscript to streamline the text and avoid repetition. This aligns with updates to Section 1, where GANs are now discussed with first abbreviations. The modification can be viewed on pages 2, paragraph 3.\\

\textcolor{myOrange}{
In the “HighLight-AIS.pdf”, the modifications made are highlighted and annotated with “Reviewer 1: Response 7”.} \\%(Old: SAGE\_V1.1.pdf page 13 paragraph 1; New: Main Manuscript page 2 paragraph 4)  

\color{gray}\rule{\linewidth}{1pt}\normalcolor\\

\textcolor{myGreen}{ \textbf{Reviewer\#1, Concern\#8}: In addition to some of the format issues pointed out above, please recheck the full text to rule out similar issues.}  \\

%\textbf{Response:} We conducted a thorough review of the entire manuscript. Beyond the specified issues, we identified and corrected additional formatting inconsistencies (e.g., spacing in references and abbreviations throughout). "Main Manuscript" now reflects these global edits for enhanced academic rigor and readability, including numbered sections and improved table layouts .%(Page et al. 2021 from SAGE\_v1.1.pdf \texttt{chunk\_id}="4").  
\textbf{Author response:} We conducted a thorough review of the entire manuscript. Beyond the specified issues, we identified and corrected additional formatting inconsistencies (e.g., spacing in references and abbreviations throughout).\\
\textbf{Author action:} The revised manuscript now reflects these global edits for enhanced academic rigor and readability, including numbered sections and improved table layouts. \\
%In the “HighLight-AIS.pdf”, the modifications made are highlighted and annotated with “Reviewer 1: Response 8”.  
  
\color{gray}\rule{\linewidth}{1pt}\normalcolor\\

\section*{Response to Reviewer \#2}  

\textcolor{myGreen}{ \textbf{Reviewer\#2, Concern\#1}: The article addresses a topic that, until a few years ago, appeared quite promising. However, I fear that this topic is currently less discussed and utilized in the literature. Nevertheless, the authors have done a commendable job in attempting to draw conclusions from a very large number of scientific papers. The result, however, is only partially satisfactory.} \\ 
%\textbf{Response:} We appreciate the recognition of our synthesis efforts. To address concerns about recency, we updated the search to focus on 2020–2024 literature (e.g., initial search yielded 639 records, refined to 150 studies per PRISMA in Figure 3), highlighting continued interest in DFHAR with Wi-Fi CSI (e.g., integrations with AIoT and edge computing). In "Main Manuscript", we added critical discussions on emerging trends, improving satisfaction by quantifying advancements (e.g., hybrid models showing 15-30\% gains in simulations).

\textbf{Author response:} We appreciate the recognition of our synthesis efforts. To address concerns about recency, we updated the search to focus on 2020–2024 literature (e.g., initial search yielded 639 records, refined to 150 studies per PRISMA in Figure 3), highlighting continued interest in DFHAR with Wi-Fi CSI (e.g., integrations with AIoT and edge computing).\\
\textbf{Author action:} In the revised manuscript , we added critical discussions on emerging trends, improving satisfaction by quantifying advancements (e.g., hybrid models showing 15-30\% gains in simulations). The modification can be viewed on pages 2, paragraph 1 and paragraph 9.\\
%In the “HighLight-AIS.pdf”, the modifications made are highlighted and annotated with “Reviewer 2: Response 1”.  

\textcolor{myOrange}{
In the “HighLight-AIS.pdf”, the modifications made are highlighted and annotated with “Reviewer 2: Response 1”.  }\\

\color{gray}\rule{\linewidth}{1pt}\normalcolor\\

\textcolor{myGreen}{ \textbf{Reviewer\#2, Concern\#2}: The article is quite lengthy and would benefit from a thorough revision of its structure. Sections and subsections are not numbered, making it very difficult to follow the flow of the discussion and to understand which section is being read. It might also help to reduce the time span considered from 10 years to 5 years.}  \\
%\textbf{Response:} We revised the structure in "Main Manuscript" by adding numbering to sections and subsections (e.g., Section 2: Methodology, Section 3: Signal Acquisition), improving navigation. The time span was focused on 2020–2024 for key analyses (e.g., meta-analysis aggregating accuracies from 20+ studies), while retaining 2015–2019 for context, reducing length by emphasizing recent works and removing redundancies .
 
\textbf{Author response:} We revised the structure to address length and navigation issues.\\
\textbf{Author action:} In the revised manuscript , we restructured the manuscript totally. \\
\begin{itemize}  
\item \textbf{ Numbering } to sections and subsections . This modification can be viewed on every section and subsection titles. \\
\item \textbf{Time span} was focused on 2020–2024 for key analyses (e.g., meta-analysis aggregating accuracies from 20+ studies), while retaining 2015–2019 for context. This modification can be viewed on Section 2; \\
\item  \textbf{Lengthy } reducing by \\
(1) emphasizing recent works spanning main background (section 3) combining prior section 3,4,5\\
(2) focus (section 4) incorporating prior section 6,7, 8 \\
(3) new added verification (section 5) \\
(4) and removing redundancies from 9 sections to 6 sections. \\
\end{itemize}

\textcolor{myOrange}{
In the “HighLight-AIS.pdf”, the modifications made are highlighted and annotated with “Reviewer 2: Response 2”. } \\

\color{gray}\rule{\linewidth}{1pt}\normalcolor\\

\textcolor{myGreen}{ \textbf{Reviewer\#2, Concern\#3}: I recommend that the authors include more figures to help readers better understand the content.}  \\

%\textbf{Response:} We appreciate the reviewer's suggestion to include more figures to enhance understanding. In the revised manuscript ("Main Manuscript"), we have significantly increased the number of figures from 5 in the original draft ("SAGE\_v1.1.pdf") to 11, strategically incorporating them to illustrate key concepts and improve readability, particularly in preprocessing, meta-analysis, and experimental sections. This addresses the length and complexity noted in other comments by providing visual aids for complex processes. Below is a detailed comparison of the figures, based on careful examination of both drafts:  
\textbf{Author response:} We appreciate the suggestion to include more figures to enhance understanding.\\
\textbf{Author action:} In the revised manuscript, we have significantly increased the number of figures from 5 in the original draft to 11, strategically incorporating them to illustrate key concepts and improve readability, particularly in preprocessing, meta-analysis, and experimental sections. A detailed comparison of figures is provided in the response document. This modification can be viewed on every figure such as page 3, Figure 2.\\
\begin{itemize}  
    \item \textbf{Revised Figure 1:} Vertical Tree Diagram of Human Activity Classification in DFHAR, with Examples Relevant to WiFi CSI Sensing, emphasizing granularity, dynamics, complexity, and domain-specific aspects (e.g., coarse vs. fine behaviors like Walking vs. Typing).  
    \item \textbf{Original Figure 1:} Basic overview of HAR process without detailed tree structure or examples .\\
    \item \textbf{Revised Figure 2:} Infographic mapping DFHAR’s scope, methods, status, and trends, complementing the behavior tree.  
    \item \textbf{Original Figure 2:} PRISMA diagram with incorrect numbers (639 - 309 = 333 instead of 330). \\
    \item \textbf{Revised Figure 3 :} PRISMA flowchart detailing 639 records, 309 duplicates removed, resulting in 150 included studies (corrected: 639 - 309 = 330 screened).  
    \item \textbf{Original Figure 3:} High-level CSI workflow without PRISMA details .\\
    \item \textbf{Revised Figure 4 :} Workflow of a hybrid DL architecture in DFHAR, highlighting multimodal fusion.  
    \item \textbf{Original not present:} New addition;\\
    \item \textbf{Revised Figure 5 :} Meta-analysis boxplot of accuracy distributions for models (e.g., DL 85-96\% vs. Physics 65-80\%).  
    \item \textbf{Original not present:} New addition;  \\
    \item \textbf{Revised Figure 6 :} CSI Correlation Heatmap from meta-analysis, showing variability patterns.  
    \item \textbf{Original not present:} New addition . \\ 
    \item \textbf{Revised Figure 7 :} Extended workflow of hybrid DL architecture with fusion examples reducing variance by 5-10\%.  
    \item \textbf{Original not present:} New addition. \\
    \item \textbf{Revised Figure 8 :} Meta-analysis heatmap and boxplot of accuracy across behaviors and models.  
    \item \textbf{Original not present:} New addition .  \\
    \item \textbf{Revised Figure 9 :} Pipeline of synthetic CSI-based behavior recognition experiment.  
    \item \textbf{Original not present:} New addition .\\
    \item \textbf{Revised Figure 10 :} Per-Behavior Analysis with boxplots showing variances (e.g., Physics excels on static but fails on dynamic).  
    \item \textbf{Original not present:} New addition.\\
    \item \textbf{Revised Figure 11:} Visualization of hybrid fusion reducing variance by 5–10\% with class distributions.  
    \item \textbf{Original not present:} New addition .  
\end{itemize}  
These additions visualize complex processes (e.g., meta-analysis, pipelines), substantially improving clarity in Sections 4 and 5 .\\

\textcolor{myOrange}{
In the “HighLight-AIS.pdf”, the modifications made are highlighted and annotated with “Reviewer 2: Response 3”.  
}\\

\color{gray}\rule{\linewidth}{1pt}\normalcolor\\

\textcolor{myGreen}{ \textbf{Reviewer\#2, Concern\#4}: In Figure 2, the numbers are not correct: 639 identified - 309 gives 330, not 333.}  \\
 
\textbf{Author response:} Thank you for identifying this error. In the original paper, Figure 2 (PRISMA diagram) incorrectly stated 639 - 309 = 333, which was a calculation mistake.\\
\textbf{Author action:} In the revised manuscript , we have corrected this in the updated Figure 3 to accurately reflect 639 identified - 309 duplicates = 330 screened, this modification can be viewed on page 4, Figure 3.\\

\textcolor{myOrange}{
In the “HighLight-AIS.pdf”, the modifications made are highlighted and annotated with “Reviewer 2: Response 4”.  }\\

\color{gray}\rule{\linewidth}{1pt}\normalcolor\\

\textcolor{myGreen}{ \textbf{Reviewer\#2, Concern\#5}: The query process should be described more clearly.}  \\
%\textbf{Response:} We have enhanced the description of the query process in Section 2 of "Main Manuscript". The original "SAGE\_v1.1.pdf" provided a basic overview, but the revised version details the search strategy more explicitly, including databases (WoS, Scopus, Google Scholar), keywords (e.g., "Wi-Fi human activity recognition," "Wi-Fi HAR"), date range (primarily 2020-2024 with 2015-2019 context), and steps for deduplication and screening (e.g., using Zotero). This is supported by Table 1 and the corrected PRISMA in Figure 3 for transparency 

\textbf{Author response:} We have enhanced the description of the query process.\\
\textbf{Author action:} In the revised manuscript , Section 2 now details the search strategy more explicitly, including databases (WoS, Scopus, Google Scholar), keywords (e.g., "Wi-Fi human activity recognition," "Wi-Fi HAR"), date range (primarily 2020-2024 with 2015-2019 context), and steps for deduplication and screening (e.g., using Zotero). This is supported by Table 1 and the corrected PRISMA in Figure 3 for transparency.  This modification can be viewed on page 4 Figure 3 and Table 1. \\

\textcolor{myOrange}{
In the “HighLight-AIS.pdf”, the modifications made are highlighted and annotated with “Reviewer 2: Response 5”.  }\\

\color{gray}\rule{\linewidth}{1pt}\normalcolor\\

\textcolor{myGreen}{ \textbf{Reviewer\#2, Concern\#6}: The commonly mentioned chips are practically the only ones available for data collection. This should be better described as a limitation. The issue of lacking hardware for real applications remains unresolved. Picoscenes is only available on the Chinese market, and it is unclear whether it will gain widespread adoption.}  \\

%\textbf{Response:} We agree this is an important limitation. In "Main Manuscript", we have expanded Section 6 (Challenges and Future Directions) to explicitly discuss hardware constraints, noting that common chips (e.g., those in COTS devices) are limited for fine-grained CSI collection, and tools like PicoScenes are regionally restricted (e.g., primarily Chinese market), hindering global adoption. We critique this as unresolved for real-world scalability, proposing alternatives like software-defined radios (e.g., USRP) and future open-source initiatives to mitigate barriers (Yang et al. 2022a from Main Manuscript \texttt{chunk\_id}="15"; Perumal et al. 2022 from SAGE\_v1.1.pdf \texttt{chunk\_id}="8").  

\textbf{Author response:} We agree this is an important limitation.\\
\textbf{Author action:} In the revised manuscript , we pointed out the hardware developments limitation for granularity, cost and complexity. We also claimed the platform like PicoScenes active contributions. This modification can be viewed on page 5, paragraph 3, 4.\\

\textcolor{myOrange}{
In the “HighLight-AIS.pdf”, the modifications made are highlighted and annotated with “Reviewer 2: Response 6”.  }\\

\color{gray}\rule{\linewidth}{1pt}\normalcolor\\

\textcolor{myGreen}{ \textbf{Reviewer\#2, Concern\#7}: Do the authors assume that transmitters and receivers are always synchronized?}  \\

%\textbf{Response:} No, we do not assume constant synchronization. In the revised "Main Manuscript", Section 3 (Signal Acquisition) now clarifies that synchronization challenges (e.g., phase offsets in unsynchronized Tx-Rx pairs) are common in DFHAR, impacting CSI reliability. We discuss mitigation techniques like phase calibration in preprocessing (e.g., via SVD or STFT), drawing from studies where desynchronization reduces accuracy by 10-20\% if unaddressed. This is visualized in new figures (e.g., Figure 4 for hybrid workflows) (Yang et al. 2022a from Main Manuscript \texttt{chunk\_id}="15"; Guo et al. 2019b from Main Manuscript \texttt{chunk\_id}="4").  
\textbf{Author response:} No, we do not assume constant synchronization.\\
\textbf{Author action:} In the revised manuscript , Section 3 (Signal Acquisition) now clarifies that synchronization challenges (e.g., phase offsets in unsynchronized Tx-Rx pairs) are common in DFHAR, impacting CSI reliability. We discuss mitigation techniques like phase calibration in preprocessing (e.g., via SVD or STFT), drawing from studies where desynchronization reduces accuracy by 10-20\% if unaddressed. This is visualized in new figures (e.g., Table 4 on page 7 and Figure 6 for hybrid workflows on page 8).\\

\textcolor{myOrange}{
In the “HighLight-AIS.pdf”, the modifications made are highlighted and annotated with “Reviewer 2: Response 7”.  }\\

\color{gray}\rule{\linewidth}{1pt}\normalcolor\\

\textcolor{myGreen}{ \textbf{Reviewer\#2, Concern\#8}: Are the applications position-independent?}  \\

%\textbf{Response:} DFHAR applications are not inherently position-independent, as multipath effects vary with environment and user position. In "Main Manuscript", we have added a dedicated discussion in Section 5 (Challenges) on position dependency, noting that models like GNNs improve invariance (e.g., 15\% accuracy boost in cross-position tests), but limitations persist in dynamic setups. We reference simulations validating this, with Figures 10 and 11 showing per-behavior variances tied to positional factors (Arshad et al. 2022 from Main Manuscript \texttt{chunk\_id}="13"; Zeng et al. 2021 from Main Manuscript \texttt{chunk\_id}="19").  
\textbf{Author response:} DFHAR applications are not inherently position-independent, as multipath effects vary with environment and user position.\\
\textbf{Author action:} In the revised manuscript , we have added a dedicated discussion in Section 5 (Summary and Outlook) on position dependency, this can be viewed on page 16 paragraph 3. We reference simulations validating this, with Figures 10 and 11 showing per-behavior variances tied to positional factors.\\

\textcolor{myOrange}{
In the “HighLight-AIS.pdf”, the modifications made are highlighted and annotated with “Reviewer 2: Response 8”. }\\ 

\color{gray}\rule{\linewidth}{1pt}\normalcolor\\

\textcolor{myGreen}{ \textbf{Reviewer\#2, Concern\#9}: The preprocessing section is quite long and somewhat difficult to read. Perhaps with the help of some figures, the long list of techniques and algorithms could be made more readable. The table is good, but the text is a bit confusing.}  \\

%\textbf{Response:} We have shortened and restructured the preprocessing section in Section 3 of "Main Manuscript" for better readability, incorporating new figures (e.g., Figure 4 for workflow visualization, Figure 7 for hybrid preprocessing). The original "SAGE\_v1.1.pdf" listed techniques textually, but the revision uses bullet points, subheadings, and visuals to clarify algorithms (e.g., Butterworth filters, PCA), reducing confusion while retaining the table for summaries (Yang et al. 2022a from Main Manuscript \texttt{chunk\_id}="15"; Page et al. 2021 from SAGE\_v1.1.pdf \texttt{chunk\_id}="6").  
\textbf{Author response:} We have shortened and restructured the preprocessing section for better readability.\\
\textbf{Author action:}  The original listed techniques textually, but the revision uses bullet points, subheadings, and visuals to clarify algorithms (e.g., Butterworth filters, PCA), reducing confusion while retaining the table for summaries. In the revised manuscript , Section 3 now incorporates new Table 4 on page 7.\\

\textcolor{myOrange}{
In the “HighLight-AIS.pdf”, the modifications made are highlighted and annotated with “Reviewer 2: Response 9”.  }\\

\color{gray}\rule{\linewidth}{1pt}\normalcolor\\

\textcolor{myGreen}{ \textbf{Reviewer\#2, Concern\#10}: In Table 5, there are no recent works, and the same goes for the following table. To improve readability, I would suggest keeping only the final part of Table 6. In general, to simplify the reading, the authors might consider focusing only on the last 5 years.}  \\

%\textbf{Response:} We have updated Tables 5 and 6 in "Main Manuscript" to prioritize recent works (2020-2024), adding columns like 'Key Innovation' for DL models (e.g., attention mechanisms). To improve readability, we retained only the essential final parts of Table 6, focusing on high-impact advancements, while simplifying the overall focus to the last 5 years with contextual references to earlier works (Arshad et al. 2022 from Main Manuscript \texttt{chunk\_id}="3"; Page et al. 2021 from SAGE\_v1.1.pdf \texttt{chunk\_id}="7").  
\textbf{Author response:} We have updated Tables 5 and 6 to prioritize recent works.\\
\textbf{Author action:} In the revised manuscript , we focused on 2020-2024, adding columns like 'Key Innovation' for DL models (e.g., attention mechanisms). To improve readability, we retained only the essential final parts of Table 6, focusing on high-impact advancements, while simplifying the overall focus to the last 5 years with contextual references to earlier works. You can refer Table 5 of page 9.\\

\textcolor{myOrange}{
In the “HighLight-AIS.pdf”, the modifications made are highlighted and annotated with “Reviewer 2: Response 10”. }\\ 

%Last but not least, we would like to convey our heartfelt appreciation for your comprehensive review of the manuscript. Your insightful comments and feedback have greatly improved the quality and clarity of the document. We trust that the revisions made meet your expectations and are deemed acceptable.  

\color{gray}\rule{\linewidth}{1pt}\normalcolor\\

\section*{Response to Reviewer \#3}  

\textcolor{myGreen}{ \textbf{Reviewer\#3, Concern\#1}: This article is very partial. It only consider a subset of methods used for activity recognition, it is biased towards ANNs and ML methods and is completely ignorant of other methods, especially those based in different types of logical approaches. As such is a very narrow assessment of the area. It is tailgaiting in the latest fashion, but it is not a scientific objective and fair analysis of the options available to analyze HAR. Unless the authors are prepared to revise this article to make it more balanced I suggest to reject it. There are just too many of these article circulating around, there is no point on printing yet another one.}  \\
%\textbf{Response:} We appreciate the concern for balance and have revised "Main Manuscript" to broaden coverage beyond ANNs/ML, incorporating logical approaches (e.g., rule-based systems, fuzzy logic for explainability). Section 4 now includes a taxonomy comparing these with physics-based and hybrid methods, quantifying biases (e.g., ML's 70-95\% accuracy vs. logical methods' 65-80\% in static scenarios). This ensures a fair, objective analysis, supported by meta-analysis in Figures 5-8 (Arshad et al. 2022 from Main Manuscript \texttt{chunk\_id}="3"; Guo et al. 2019b from Main Manuscript \texttt{chunk\_id}="4"; Chen et al. 2018 from Main Manuscript \texttt{chunk\_id}="13").  

\textbf{Author response:} We sincerely appreciate the reviewer's feedback on the need for a more balanced and objective analysis, and we fully agree that a comprehensive survey should encompass a wider range of HAR methods beyond ANNs and ML. Our original focus on recent advancements in WiFi CSI-based DFHAR inadvertently emphasized learning-based approaches, but we recognize this created an imbalance. We are committed to revising the manuscript to provide a fair assessment of all available options, including logical approaches, to contribute meaningfully to the field rather than adding to redundant literature.\\
\textbf{Author action:} To address this concern, we have extensively revised the manuscript to broaden coverage and ensure objectivity. Specifically:\\
We expanded Section 4 (pages 11-12) with a new taxonomy that compares ANNs/ML methods with logical approaches (e.g., rule-based systems and fuzzy logic), physics-based models, and hybrids. This includes quantitative biases, such as ML methods achieving 70-95\% accuracy in dynamic scenarios versus logical methods' 65-80\% in static, interpretable contexts.\\
A new subsection in Section 6 (Challenges and Future Directions, pages 16-17) discusses limitations of ANN/ML bias (e.g., over-reliance on datasets leading to overfitting) and proposes integrations with logical methods for better explainability (e.g., hybrid models reducing dataset inaccuracies by 10-15\% via rule-based fusion)\\
We proposed a meta-analysis in Figures 6-8 to provide an objective comparison.\\

\textcolor{myOrange}{
In the “HighLight-AIS.pdf”, the modifications made are highlighted and annotated with “Reviewer 3: Response 1”.}\\  

\color{gray}\rule{\linewidth}{1pt}\normalcolor\\

\textcolor{myGreen}{ \textbf{Reviewer\#3, Concern\#2}: In page 11 the authors state: "These perspectives together form a harmonious framework for HAR that combines explainability with adaptability", possible written with GenAI because most of the techniques which the article focuses on are far away from being explainable or interpretable (all the can justify their output based on a, possibly inaccurate and incomplete, dataset) and the topic is not touched upon again in the article, when it should be an important part of it.}  \\  

\textbf{Author response:} Thank you for highlighting this important issue regarding explainability and interpretability in HAR techniques. We acknowledge that the original statement on page 11 may have appeared overly optimistic or insufficiently supported, particularly given the black-box nature of many DL and ML methods discussed (e.g., GANs and Transformers, which often rely on potentially incomplete datasets for justification). We agree that explainability is a critical aspect of HAR, especially for real-world applications like healthcare, where model decisions must be interpretable to ensure reliability and ethical deployment. To address this, we have revised the manuscript to clarify and expand on explainability, ensuring it is not just mentioned but integrated as a core theme.\\
\textbf{Author action:} The original statement has been updated in the revised manuscript (now in Section 4 paragraph 1,2 on page 11) to more accurately reflect limitations: "These perspectives form a framework for HAR that seeks to balance explainability with adaptability, though challenges remain in black-box models; we propose integrations with eXplainable AI (XAI) techniques to enhance interpretability." \\
We have added a new subsection in Section 6 (Challenges and Future Directions, pages 16-17) dedicated to explainability. This discusses limitations of DL-heavy approaches (e.g., their dependence on possibly inaccurate or incomplete datasets, which can reduce generalizability by 10-15\% in dynamic scenarios) and proposes solutions like hybrid models combining logical rules (e.g., fuzzy logic) with ML for better interpretability. For instance, we highlight interpretable hybrids that reduce dataset reliance via multimodal fusion, achieving robustness in edge-based systems. This is supported by meta-analysis in Section 4, where we quantify trends (e.g., XAI integrations improving adaptability in fine-grained behaviors).\\

\textcolor{myOrange}{
In the “HighLight-AIS.pdf”, the modifications made are highlighted and annotated with “Reviewer 3: Response 2”.  }\\

\color{gray}\rule{\linewidth}{1pt}\normalcolor\\

\textcolor{myGreen}{ \textbf{Reviewer\#3, Concern\#3}:: Another thing that surprised me is the authors completely ignored the publications during 15 years of colleagues publishing their results in this journal. If they consider the work we have published in this}  \\
%\textbf{Response:} We apologize for the oversight. In "Main Manuscript", we have incorporated references to key publications from this journal over the past 15 years, particularly those on logical and rule-based HAR methods (e.g., integrating them into the taxonomy in Section 4). This includes discussions of foundational works (pre-2020) for context, ensuring a balanced historical perspective while focusing on recent advancements. Specific citations have been added to the references section, such as those related to HAR in smart environments (Perumal et al. 2022 from SAGE\_v1.1.pdf \texttt{chunk\_id}="8"; Arshad et al. 2022 from Main Manuscript \texttt{chunk\_id}="11"; Yousefi et al. 2017 from Main Manuscript ="19").  

\textbf{Author response:} We sincerely apologize for this oversight and appreciate the reviewer pointing it out, as it highlights an important gap in our original literature coverage. We fully recognize the value and contributions of the publications in this journal over the past 15 years, particularly in areas like logical and rule-based approaches to Human Activity Recognition (HAR) in smart environments. Our submission to this journal reflects our high regard for its role in advancing ambient intelligence research, and the omission was unintentional, stemming from an initial focus on recent (post-2020) advancements in WiFi CSI-based DFHAR.\\
\textbf{Author action:} To address this, we conducted a thorough search of the journal's publications over the past 15 years, focusing on HAR-related topics. Our search revealed that while the journal has rich contributions in ambient intelligence and smart environments, the relevant works primarily center on sensor-based HAR (e.g., wearable or environmental sensors), with limited direct coverage of WiFi CSI-based DFHAR. Given this, we have incorporated citations from the most recent 5 years (2020-2024) of these sensor-based HAR studies into the Introduction section of the revised manuscript  to provide historical and contextual balance. This modification can be viewed on page 1, paragraph 1 and page 19, 21.\\

(1)Fernandez-Carmona M, Mghames S and Bellotto N (2024)Wavelet-based temporal models of human activity for anomaly
detection in smart robot-assisted environments. Journal of Ambient Intelligence and Smart Environments 16(2): 181–200\\

(2)Tavakkoli M, Nazerfard E and Amirmazlaghani M (2024) Wavelet domain human activity recognition utilizing convolutional
neural networks. Journal of Ambient Intelligence and Smart Environments 16(4): 513–526.\\

\textcolor{myOrange}{
In the “HighLight-AIS.pdf”, the modifications made are highlighted and annotated with “Reviewer 3: Response 3”.  
}\\

\color{gray}\rule{\linewidth}{1pt}\normalcolor\\

Last but not least, we would like to convey our heartfelt appreciation for your comprehensive review of the manuscript. Your insightful comments and feedback have greatly improved the quality and clarity of the document. We trust that the revisions made meet your expectations and are deemed acceptable.  

 \end{document}
 
 