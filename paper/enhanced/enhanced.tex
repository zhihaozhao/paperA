% !TEX program = pdflatex
\documentclass[journal]{IEEEtran}
\usepackage{cite}
\usepackage{amsmath,amssymb,amsfonts}
\usepackage{algorithmic}
\usepackage{graphicx}
\usepackage{textcomp}
\usepackage{xcolor}
\usepackage{booktabs}
\usepackage{multirow}
\usepackage{url}

\def\BibTeX{{\rm B\kern-.05em{\sc i\kern-.025em b}\kern-.08em
    T\kern-.1667em\lower.7ex\hbox{E}\kern-.125emX}}

\begin{document}

\title{PINN-like Enhanced Architecture for WiFi CSI Sensing: CNN + SE + Temporal Attention with Calibrated and Interpretable Sim2Real Performance}

\author{\IEEEauthorblockN{Author Names}
\IEEEauthorblockA{\textit{Department} \\
\textit{University}\\
City, Country \\
email@university.edu}}

\maketitle

\begin{abstract}
We investigate a PINN-inspired Enhanced architecture for WiFi Channel State Information (CSI) human activity recognition (HAR) that integrates convolutional feature extraction, squeeze-and-excitation (SE) channel attention, and temporal attention, paired with calibrated inference. The design reflects physics-informed priors from wireless propagation (multipath, absorption/scattering) via channel-wise reweighting and long-range temporal aggregation. Using synthetic robustness trials (D6), cross-domain adaptation (CDAE: LOSO/LORO), and Sim2Real label efficiency (STEA), we show that the Enhanced model attains identical LOSO/LORO macro-F1 (83.0±0.1\%) and reaches 82.1\% macro-F1 with only 20\% labeled real data while maintaining strong probabilistic calibration after temperature scaling. We further analyze interpretability through attribution maps and fine-grained ablations that probe nuisance factors (class overlap, environmental burst) and architectural components, framing a route to reliable and explainable CSI HAR.
\end{abstract}

\begin{IEEEkeywords}
WiFi CSI, Human Activity Recognition, Squeeze-and-Excitation, Temporal Attention, PINN-inspired design, Calibration, Explainability
\end{IEEEkeywords}

\section{Introduction}
CSI-based sensing is a compelling alternative to camera or wearable systems, yet practical deployment hinges on robust generalization, calibrated probabilities, and credible explanations. Benchmarks such as SenseFi~\cite{yang2023sensefi} consolidate supervised performance trends across datasets and architectures, but real deployments often confront domain shift and limited labels. These realities motivate architectures that are not only accurate but also stable across domains and transparent about uncertainty.

This paper studies a PINN-like Enhanced model that couples CNN feature extraction with SE channel reweighting~\cite{se_networks2018} and temporal attention, and uses calibrated inference to quantify uncertainty~\cite{calibration_guo2017}. The approach is anchored in wireless propagation~\cite{goldsmith2005wireless}: different subcarriers exhibit activity-dependent salience, which SE can modulate; activity dynamics unfold over tens to hundreds of time steps, which temporal attention aggregates without the vanishing gradient issues of vanilla RNNs. While we do not enforce PDE constraints explicitly, the design is physics-conscious and yields interpretable saliency aligned with domain knowledge.

Our contributions are threefold. First, we present an Enhanced architecture that achieves strong performance and trustworthy calibration in synthetic robustness trials and cross-domain settings. Second, we conduct Sim2Real label-efficiency analyses showing 82.1\% macro-F1 at 20\% labels (98.6\% of full 83.3\%), demonstrating practical annotation savings. Third, we provide interpretability and ablations, including attribution maps and nuisance-factor sweeps, to illuminate how and why the model behaves reliably.

\textbf{Key Contributions}
\begin{enumerate}
  \item \textbf{PINN-like Enhanced model:} CNN + SE + temporal attention architecture grounded in propagation-informed inductive biases with calibrated inference.
  \item \textbf{Trustworthy evaluation:} Accuracy plus calibration (ECE/NLL/Brier) across synthetic and cross-domain regimes, demonstrating reliable probabilities for IoT decision thresholds.
  \item \textbf{Label efficiency:} Sim2Real STEA shows 82.1\% macro-F1 at 20\% labels, nearing 83.3\% full-supervision with 80\% annotation savings.
  \item \textbf{Interpretability and ablation:} Attribution maps and nuisance-factor sweeps reveal stable reliance on physically meaningful subcarriers and temporal spans.
\end{enumerate}

The remainder of this paper is organized as follows. Section II reviews related work in CSI sensing, attention/SE architectures, and calibration. Section III details the Enhanced architecture and PINN-like perspective. Section IV describes experimental protocols for synthetic robustness, CDAE, and STEA. Section V presents quantitative results, ablations, and attribution. Section VI discusses implications and limitations, and Section VII concludes.

\section{Related Work}
\subsection{WiFi CSI human sensing}
Early device-free sensing relied on handcrafted features over amplitude/phase dynamics, Doppler signatures, and path length surrogates. With the advent of public datasets and toolchains, deep learning replaced manual features and improved abstraction capacity. SenseFi~\cite{yang2023sensefi} surveyed 11 models on 4 datasets, reporting wide variation across tasks and protocols and emphasizing the need for systematic evaluation. Subsequent studies layered attention over CNN or RNN backbones to better model temporal dependencies, yet challenges persist: domain shift across subjects/environments, ill-calibrated probabilities, and label scarcity at deployment time.

\subsection{Attention and channel reweighting}
Temporal attention has emerged as a robust mechanism for long-range sequence modeling in action recognition and time-series forecasting~\cite{li2020tea,bertasius2021timesformer,lim2021tft,zhou2021informer}. In parallel, squeeze-and-excitation (SE)~\cite{se_networks2018} demonstrated that channel-wise reweighting can significantly improve representational quality by highlighting informative channels while suppressing noise. In CSI, subcarriers and antenna combinations respond differently to human motion and multipath; SE is therefore a natural fit, enabling adaptive emphasis aligned with propagation structure.

\subsection{Sim2Real and calibration}
Simulation-to-reality transfer is established in robotics via domain randomization~\cite{peng2018sim2real}, and is gaining traction in sensing where real data collection is costly. However, high accuracy alone is insufficient in IoT; predictions must be \emph{calibrated}. Guo et al.~\cite{calibration_guo2017} formalized calibration metrics and temperature scaling, which we adopt to ensure that confidence reflects correctness. Our work brings these strands together: a physics-conscious architecture, synthetic diversity for robustness, and trustworthy evaluation beyond accuracy.

\section{Enhanced Architecture and PINN-like Perspective}
The Enhanced model composes three components: (i) convolutional layers for local spatiotemporal filtering of CSI tensors, (ii) SE attention to adaptively emphasize subcarrier/antenna channels implicated by multipath structure~\cite{se_networks2018}, and (iii) temporal attention to aggregate long-range activity patterns. While not enforcing PDE constraints explicitly, the design is \emph{PINN-like} in spirit: architectural choices reflect inductive biases anchored in propagation phenomena, which we find to support stable training and calibrated inference.

\subsection{Mathematical outline}
Let $\mathbf{X}\in \mathbb{R}^{T\times F}$ denote a CSI window with $T$ time steps and $F$ frequency/subcarrier features (optionally stacked over antenna pairs). CNN blocks compute local features $\mathbf{H}^{(\ell)} = \mathrm{Conv}^{(\ell)}(\mathbf{H}^{(\ell-1)})$ with residual connections. SE modules perform channel reweighting $\tilde{\mathbf{H}} = \mathbf{H}\odot \sigma(\mathbf{W}_2\,\delta(\mathbf{W}_1\,\mathrm{GAP}(\mathbf{H})))$, where GAP is global average pooling across time, $\delta$ is ReLU, and $\sigma$ is sigmoid. Temporal attention computes weights $\alpha_t \propto \exp(\mathbf{w}^\top \tanh(\mathbf{W}\,\tilde{\mathbf{h}}_t))$ and aggregates $\mathbf{c}=\sum_t \alpha_t\tilde{\mathbf{h}}_t$. A classifier maps $\mathbf{c}$ to logits, which are temperature-scaled at inference: $\mathbf{z}'=\mathbf{z}/T$ with $T>0$ fit on validation to minimize NLL~\cite{calibration_guo2017}.

\subsection{Complexity and capacity alignment}
To ensure fair comparison under CDAE/STEA, we align parameter counts within ±10\% across Enhanced, CNN, BiLSTM, and Conformer-lite. We use identical optimization settings (epochs, batch, learning rate schedules) and evaluate multiple seeds, reporting mean±std for macro-F1 and calibration metrics. This controls for capacity-driven confounds and isolates architectural contributions (SE, attention, calibration).

\section{Experimental Setup}
We evaluate in three regimes: (1) synthetic robustness (D6) to stress calibration and accuracy; (2) cross-domain adaptation (CDAE: LOSO/LORO) to test domain-agnostic features; and (3) Sim2Real label efficiency (STEA) to quantify annotation budgets. Models are capacity-aligned to ensure fair comparison. Calibration uses temperature scaling tuned on validation to minimize NLL. All metrics (macro-F1, ECE, NLL, Brier) are extracted from the JSON logs in `results\_gpu`.

\begin{figure}[t]
\centering
\includegraphics[width=\columnwidth]{plots/d6_calibration_summary.pdf}
\caption{D6 synthetic robustness: macro-F1 and ECE (mean\,\textpm\,std) across models. Enhanced attains strong accuracy with improved calibration after temperature scaling.}
\label{fig:d6_cal}
\end{figure}

\section{Results: Performance and Reliability}
Figure~\ref{fig:d6_cal} summarizes D6 experiments: Enhanced exhibits macro-F1 near unity on hard synthetic regimes with markedly reduced ECE after calibration, compared to CNN, BiLSTM, and Conformer-lite. High accuracy \emph{plus} honest probabilities is essential for downstream thresholds, triage, or selective abstention in IoT.

\begin{figure}[t]
\centering
\includegraphics[width=\columnwidth]{plots/d5_progressive_enhanced.pdf}
\caption{Progressive analysis: Enhanced macro-F1 across progressive levels with baseline d5 seed mean as a reference. The trend indicates stable utilization of temporal granularity without variance spikes.}
\label{fig:d5_prog}
\end{figure}

Figure~\ref{fig:d5_prog} shows stable accuracy as temporal granularity varies, consistent with the hypothesis that temporal attention captures long-range patterns while SE focuses channel responses relevant to multipath-induced structure. We observe small variance and no collapse at coarser levels, indicating robust temporal aggregation.

\begin{figure}[t]
\centering
\includegraphics[width=\columnwidth]{plots/ablation_noise_env.pdf}
\caption{Ablation heatmaps (D2): macro-F1 vs. class overlap (y) and env burst (x) with fixed label noise. Enhanced remains robust as nuisance factors increase.}
\label{fig:ablation_d2}
\end{figure}

\begin{figure}[t]
\centering
\includegraphics[width=\columnwidth]{plots/ablation_components.pdf}
\caption{Component-level comparison on D6: Enhanced vs. capacity-aligned CNN, BiLSTM, and Conformer-lite baselines.}
\label{fig:ablation_components}
\end{figure}

Fine-grained ablations in Figure~\ref{fig:ablation_d2} probe nuisance factors (class overlap, env burst). Enhanced sustains high macro-F1 across challenging corners, while baselines degrade more sharply. Component-level comparisons in Figure~\ref{fig:ablation_components} position Enhanced against capacity-aligned baselines, emphasizing that SE and temporal attention are complementary rather than interchangeable.

\section{Interpretability: Attribution and Physics Cues}
We probe interpretability with Grad-CAM~\cite{selvaraju2017gradcam} and Integrated Gradients~\cite{sundararajan2017ig}. On CSI tensors, these methods highlight subcarriers and temporal spans driving predictions. Enhanced concentrates attribution on coherent subcarrier bands and motion-aligned windows, aligning with multipath intuition. While attribution is not a guarantee, it offers a transparent lens into decision pathways and helps identify potential shortcuts or spurious activations.

\begin{figure}[t]
\centering
\includegraphics[width=\columnwidth]{plots/attribution_examples.pdf}
\caption{Attribution maps (illustrative layout): banded subcarrier salience and localized temporal focus are consistent with propagation-informed expectations.}
\label{fig:attribution}
\end{figure}

\section{Discussion}
Our findings echo benchmark conclusions~\cite{yang2023sensefi} that attention mechanisms improve CSI HAR, while adding that SE-based channel reweighting and calibrated inference jointly stabilize predictions. The PINN-like framing bridges physical priors and deep architectures, guiding design without strict PDE supervision. Limitations include scope of interpretability case studies and reliance on post-hoc calibration; future work should integrate domain-aware calibration and physics-constrained objectives more tightly and evaluate complex, multi-person settings.

\section{Conclusion}
We presented a PINN-inspired Enhanced architecture that combines CNN, SE, and temporal attention, demonstrating strong performance and improved calibration on synthetic robustness trials, with interpretable attribution patterns consistent with propagation insights. The results support a practical path toward reliable, explainable CSI sensing.

\bibliographystyle{IEEEtran}
\bibliography{enhanced_refs}

\end{document}