% 综述章节:与实验章节联动,基准化与量化合成
\chapter{WiFi CSI 人体活动识别综述:数据—方法—评测的基准化与量化合成}
\label{chap:review}

本章面向博士论文的整体目标,基于 \texttt{benchmarks/sage\_latex\_DFHAR} 的综述与现有实验章节(\chapref{chap:experiments}),形成系统性的“数据—方法—评测”框架:
(1) 数据层:公开数据集、合成数据、跨域划分与元数据标准;
(2) 方法层:RSSI/CSI 信号建模、时频图像化、深度模型与物理约束;
(3) 评测层:统一任务定义、指标与协议,面向泛化与可信的基准化评测。

\section{数据层:公开与合成的统一描述}
概述公开 HAR/DFHAR 数据集的采集条件、天线/子载波配置、活动集合与标注质量;给出合成数据生成与域映射策略,并与实验章节中的合成管线对齐,强调跨域划分(LOSO/LORO)与元数据表头的统一。

\section{方法层:从信号到表示与模型}
系统梳理 RSSI/CSI 的物理建模、预处理(去噪、子载波选择)、时频变换(STFT/CWT)、图像化表示与深度模型(CNN/LSTM/Transformer),并与实验章节中的模型配置联动,突出物理先验与正则化对泛化的作用。

\section{评测层:统一任务、指标与协议}
定义识别、分段、行为序列建模等任务;统一 Top-1/Top-5、F1、宏/微平均、ECE/校准等指标;提出跨主体/场景/设备的协议,并提供再现实验清单与脚本入口,支持 \texttt{reproducibility} 一键复现。

\section{DFHAR 实验验证与开源资源}
为增强综述与实验的闭环,我们采用开源仓库 \url{https://github.com/zhihaozhao/DFHAR} 中的真实实验作为再现依据,包含:
(1)数据与脚本:DFHAR 的数据组织、预处理与评测脚本;
(2)统一指标:与本章评测层一致的 Top-1/F1/ECE 等指标输出;
(3)图表生成:箱线图、热力图、气泡图生成脚本与示例配置;
(4)可获得性:版本化发布并可与论文 \texttt{reproducibility} 目录联动。
为便于读者复用,我们在附录提供脚本清单与运行说明,并在主仓库提供 DOI/Release 链接。

\section{跨研究差距与路线图}
从 2015–2024 的代表性工作中归纳差距:
(1) 非结构化场景鲁棒性与域外泛化不足;
(2) 高计算/低功耗约束下的边缘部署缺口;
(3) 基准与复现资源分散;
(4) 与安全/隐私/伦理的协同不足。提出短中长期路线图,与实验章节的系统实现形成闭环。

\section{小结}
本章与实验章节共同构成从“问题—方法—系统—评测”的闭环:综述提供标准与基准,实验落实实现与验证,二者相互校验、相互促进。
