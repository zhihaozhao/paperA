% 附录B:实验配置详细信息
\chapter{实验配置详细信息}
\label{app:experiment_details}

本附录提供了主文实验章节中所有实验配置的详细技术参数、硬件环境和软件版本信息,确保实验的完全可重现性。

\section{硬件环境配置}
\label{app:hardware_config}

\subsection{实验设备清单}
\label{app:hardware_list}

\begin{table}[h]
\centering
\caption{主要实验设备配置}
\label{tab:hardware_specs}
\begin{tabular}{lll}
\toprule
\textbf{设备类型} & \textbf{型号/规格} & \textbf{用途} \\
\midrule
WiFi路由器 & TP-Link AC1900 (Archer A9) & CSI数据采集 \\
接收设备 & Intel AX200 WiFi 6网卡 & CSI信号接收 \\
计算服务器 & NVIDIA RTX 4090, 64GB RAM & 深度学习训练 \\
CPU服务器 & Intel i9-12900K, 128GB RAM & 数据预处理 \\
存储设备 & 2TB NVMe SSD & 数据存储 \\
\bottomrule
\end{tabular}
\end{table}

\subsection{网络配置参数}
\label{app:network_params}

\begin{itemize}
\item \textbf{频段}: 5GHz (802.11ac)
\item \textbf{带宽}: 80MHz  
\item \textbf{子载波数}: 256 (有效114个)
\item \textbf{天线配置}: 3×3 MIMO
\item \textbf{采样率}: 1000 Hz
\item \textbf{数据格式}: 复数矩阵 (114×3×3)
\end{itemize}

\section{软件环境配置}
\label{app:software_config}

\subsection{开发环境}
\label{app:dev_environment}

\begin{table}[h]
\centering
\caption{软件环境版本信息}
\label{tab:software_versions}
\begin{tabular}{ll}
\toprule
\textbf{软件/库} & \textbf{版本} \\
\midrule
Python & 3.10.12 \\
PyTorch & 2.0.1+cu118 \\
NumPy & 1.24.3 \\
SciPy & 1.11.1 \\
Matplotlib & 3.7.2 \\
Seaborn & 0.12.2 \\
Pandas & 2.0.3 \\
Scikit-learn & 1.3.0 \\
CUDA & 11.8 \\
\bottomrule
\end{tabular}
\end{table}

\subsection{依赖库详细配置}
\label{app:dependencies}

\begin{lstlisting}[language=bash, caption=环境配置脚本]
# 创建Python环境
conda create -n wifi_csi_phd python=3.10
conda activate wifi_csi_phd

# 安装深度学习框架
pip install torch==2.0.1+cu118 torchvision==0.15.2+cu118 \
    --index-url https://download.pytorch.org/whl/cu118

# 安装科学计算库
pip install numpy==1.24.3 scipy==1.11.1 matplotlib==3.7.2 \
    seaborn==0.12.2 pandas==2.0.3 scikit-learn==1.3.0

# 安装评估和可视化库
pip install tensorboard==2.13.0 wandb==0.15.5 plotly==5.15.0

# 安装开发工具
pip install jupyter==1.0.0 ipykernel==6.24.0 black==23.7.0
\end{lstlisting}

\section{实验协议详细参数}
\label{app:protocol_details}

\subsection{D2协议:合成数据鲁棒性验证}
\label{app:d2_protocol}

\begin{table}[h]
\centering
\caption{D2协议详细配置参数}
\label{tab:d2_config_details}
\begin{tabular}{lll}
\toprule
\textbf{参数类别} & \textbf{参数名称} & \textbf{取值范围} \\
\midrule
\multirow{4}{*}{信号参数} 
& 信噪比 (SNR) & [5, 10, 15, 20, 25, 30] dB \\
& 多径数量 & [3, 5, 8, 12, 15] \\
& 载波频率偏移 & [-50, -25, 0, 25, 50] Hz \\
& 相位噪声 & [0.01, 0.05, 0.1, 0.2] rad \\
\midrule
\multirow{3}{*}{活动参数}
& 活动重叠度 & [0, 0.1, 0.3, 0.5, 0.7] \\
& 动作幅度 & [0.5, 1.0, 1.5, 2.0] m \\
& 持续时间 & [2, 5, 10, 15] s \\
\midrule
\multirow{3}{*}{环境参数}
& 房间大小 & [3×3, 5×5, 8×8, 10×10] m² \\
& 障碍物密度 & [0, 0.1, 0.3, 0.5] \\
& 墙体材质 & [木质, 混凝土, 金属] \\
\bottomrule
\end{tabular}
\end{table}

总计540种配置组合 = 6×5×5×4 × 5×4×4 × 4×4×3

\subsection{CDAE协议:跨域适应评估}
\label{app:cdae_protocol}

\begin{table}[h]
\centering
\caption{CDAE协议跨域配置}
\label{tab:cdae_domains}
\begin{tabular}{llll}
\toprule
\textbf{域类型} & \textbf{配置数量} & \textbf{变量} & \textbf{控制条件} \\
\midrule
LOSO & 8 & 受试者ID & 固定房间+设备 \\
LORO & 5 & 房间ID & 固定受试者+设备 \\
\bottomrule
\end{tabular}
\end{table}

总计40种配置 = 8×5种LOSO配置

\subsection{STEA协议:Sim2Real迁移效率}
\label{app:stea_protocol}

\begin{table}[h]
\centering
\caption{STEA协议标签比例配置}
\label{tab:stea_label_ratios}
\begin{tabular}{lll}
\toprule
\textbf{标签比例} & \textbf{样本数量} & \textbf{验证轮次} \\
\midrule
1\% & 约50样本 & 8轮 \\
5\% & 约250样本 & 8轮 \\
10\% & 约500样本 & 8轮 \\
20\% & 约1000样本 & 8轮 \\
50\% & 约2500样本 & 8轮 \\
100\% & 约5000样本 & 8轮 \\
\bottomrule
\end{tabular}
\end{table}

总计56种配置 = 6种标签比例 × 8轮验证 + 额外控制实验

\section{数据集详细信息}
\label{app:dataset_details}

\subsection{合成数据集规格}
\label{app:synthetic_specs}

\begin{table}[h]
\centering
\caption{合成数据集详细规格}
\label{tab:synthetic_dataset_specs}
\begin{tabular}{ll}
\toprule
\textbf{参数} & \textbf{值} \\
\midrule
总样本数量 & 50,000 \\
活动类别 & 4类 (坐立、站立、行走、跌倒) \\
序列长度 & 1000时间步 (1秒 @ 1kHz) \\
CSI矩阵维度 & 114×3×3 (子载波×天线) \\
数据类型 & 复数 (float32) \\
标注格式 & One-hot编码 \\
划分比例 & 训练60\% / 验证20\% / 测试20\% \\
\bottomrule
\end{tabular}
\end{table}

\subsection{真实数据集规格}
\label{app:real_specs}

\begin{table}[h]
\centering
\caption{真实数据集详细规格}
\label{tab:real_dataset_specs}
\begin{tabular}{ll}
\toprule
\textbf{参数} & \textbf{值} \\
\midrule
总样本数量 & 12,500 \\
受试者数量 & 8人 (年龄22-65岁) \\
房间数量 & 5个 (3×3m到8×8m) \\
采集时长 & 总计125小时 \\
活动重复次数 & 每人每活动50次 \\
标注精度 & 专家标注,双重验证 \\
\bottomrule
\end{tabular}
\end{table}

\section{训练超参数详细配置}
\label{app:hyperparameters}

\subsection{Enhanced模型超参数}
\label{app:enhanced_hyperparams}

\begin{table}[h]
\centering
\caption{Enhanced模型详细超参数}
\label{tab:enhanced_hyperparams}
\begin{tabular}{llll}
\toprule
\textbf{模块} & \textbf{参数名} & \textbf{值} & \textbf{说明} \\
\midrule
\multirow{4}{*}{卷积层}
& 卷积核数量 & [64, 128, 256] & 逐层递增 \\
& 卷积核大小 & [7, 5, 3] & 逐层递减 \\
& 步长 & [2, 1, 1] & 首层降采样 \\
& 激活函数 & ReLU & 标准激活 \\
\midrule
\multirow{3}{*}{SE模块}
& 压缩比 & 16 & 通道注意力 \\
& 激活函数 & Sigmoid & 门控机制 \\
& 位置 & 每个卷积块后 & 特征重标定 \\
\midrule
\multirow{4}{*}{BiLSTM}
& 隐藏单元数 & 128 & 双向各128 \\
& 层数 & 2 & 深度建模 \\
& Dropout & 0.3 & 防止过拟合 \\
& 序列长度 & 100 & 时间窗口 \\
\midrule
\multirow{3}{*}{时序注意力}
& 注意力头数 & 8 & 多头机制 \\
& 嵌入维度 & 256 & 与BiLSTM匹配 \\
& 位置编码 & 正弦余弦 & 相对位置 \\
\bottomrule
\end{tabular}
\end{table}

\subsection{训练配置参数}
\label{app:training_config}

\begin{table}[h]
\centering
\caption{训练过程详细配置}
\label{tab:training_config_details}
\begin{tabular}{ll}
\toprule
\textbf{参数} & \textbf{值} \\
\midrule
批处理大小 & 64 \\
初始学习率 & 1e-3 \\
学习率调度 & 余弦退火 (CosineAnnealingLR) \\
优化器 & Adam (β₁=0.9, β₂=0.999) \\
权重衰减 & 1e-4 \\
最大训练轮次 & 200 \\
早停耐心值 & 20轮 \\
验证频率 & 每5轮 \\
损失函数 & 交叉熵 + 置信度正则化 \\
正则化系数λ & [1e-4, 1e-3, 1e-2] (网格搜索) \\
\bottomrule
\end{tabular}
\end{table}

\section{评估指标计算详细公式}
\label{app:metrics_formulas}

\subsection{分类性能指标}
\label{app:classification_metrics}

\textbf{宏平均F1分数}:
\begin{equation}
\text{Macro-F1} = \frac{1}{C} \sum_{c=1}^{C} \frac{2 \cdot \text{Precision}_c \cdot \text{Recall}_c}{\text{Precision}_c + \text{Recall}_c}
\label{eq:macro_f1_detailed}
\end{equation}

其中,对于类别$c$:
\begin{align}
\text{Precision}_c &= \frac{TP_c}{TP_c + FP_c} \\
\text{Recall}_c &= \frac{TP_c}{TP_c + FN_c}
\end{align}

\textbf{加权平均F1分数}:
\begin{equation}
\text{Weighted-F1} = \sum_{c=1}^{C} w_c \cdot \text{F1}_c, \quad w_c = \frac{n_c}{N}
\label{eq:weighted_f1}
\end{equation}

\subsection{校准性能指标}
\label{app:calibration_metrics}

\textbf{期望校准误差 (ECE)}:
\begin{equation}
\text{ECE} = \sum_{m=1}^{M} \frac{|B_m|}{N} \left| \text{acc}(B_m) - \text{conf}(B_m) \right|
\label{eq:ece_detailed}
\end{equation}

其中:
\begin{itemize}
\item $B_m$:第$m$个置信度区间的样本集合
\item $\text{acc}(B_m)$:区间内的实际准确率
\item $\text{conf}(B_m)$:区间内的平均置信度
\end{itemize}

\textbf{Brier分数}:
\begin{equation}
\text{Brier} = \frac{1}{N} \sum_{i=1}^{N} \sum_{j=1}^{C} (p_{ij} - y_{ij})^2
\label{eq:brier_detailed}
\end{equation}

\section{统计检验方法}
\label{app:statistical_tests}

\subsection{配对t检验}
\label{app:paired_ttest}

对于模型性能对比,采用配对t检验验证差异显著性:

\begin{equation}
t = \frac{\bar{d}}{s_d / \sqrt{n}}
\label{eq:paired_ttest}
\end{equation}

其中:
\begin{itemize}
\item $\bar{d}$:成对差异的均值
\item $s_d$:成对差异的标准差  
\item $n$:配对样本数量
\end{itemize}

显著性水平设为$\alpha = 0.05$,并报告Cohen's d效应量。

\subsection{Bootstrap置信区间}
\label{app:bootstrap_ci}

采用1000次Bootstrap重采样构建95\%置信区间:

\begin{equation}
\text{CI}_{95\%} = [\text{Percentile}_{2.5\%}, \text{Percentile}_{97.5\%}]
\label{eq:bootstrap_ci}
\end{equation}

\section{可重现性保证措施}
\label{app:reproducibility}

\subsection{随机种子管理}
\label{app:random_seeds}

\begin{lstlisting}[language=python, caption=随机种子设置]
import torch
import numpy as np
import random

def set_reproducible_seeds(seed=42):
    """设置所有随机种子确保可重现性"""
    torch.manual_seed(seed)
    torch.cuda.manual_seed(seed)
    torch.cuda.manual_seed_all(seed)
    np.random.seed(seed)
    random.seed(seed)
    
    # 确定性算法设置
    torch.backends.cudnn.deterministic = True
    torch.backends.cudnn.benchmark = False
\end{lstlisting}

所有实验使用种子集合:[42, 123, 456, 789, 1024, 2048, 4096, 8192]

\subsection{版本控制和实验跟踪}
\label{app:version_control}

\begin{itemize}
\item \textbf{Git分支管理}:每个实验阶段使用独立分支
\item \textbf{实验标记}:每次运行生成唯一时间戳标识
\item \textbf{配置保存}:自动保存所有超参数到JSON文件
\item \textbf{结果归档}:结果文件包含完整的环境和配置信息
\end{itemize}

\section{计算资源消耗统计}
\label{app:computational_costs}

\subsection{训练时间统计}
\label{app:training_time}

\begin{table}[h]
\centering
\caption{各模型训练时间对比}
\label{tab:training_time_comparison}
\begin{tabular}{llll}
\toprule
\textbf{模型} & \textbf{GPU训练时间} & \textbf{CPU训练时间} & \textbf{内存占用} \\
\midrule
Enhanced & 45分钟 & 8小时 & 6.2 GB \\
CNN & 25分钟 & 4小时 & 3.8 GB \\
BiLSTM & 65分钟 & 12小时 & 4.5 GB \\
Conformer-lite & 85分钟 & 15小时 & 8.9 GB \\
\bottomrule
\end{tabular}
\end{table}

\subsection{存储需求}
\label{app:storage_requirements}

\begin{itemize}
\item \textbf{合成数据}:约120 GB (50,000样本 × 2.4MB/样本)
\item \textbf{真实数据}:约30 GB (12,500样本 × 2.4MB/样本)  
\item \textbf{模型检查点}:约2 GB (每个模型50MB × 40种配置)
\item \textbf{实验结果}:约500 MB (CSV/JSON文件)
\item \textbf{图表输出}:约100 MB (PDF/PNG文件)
\end{itemize}

总存储需求:约152.6 GB

\section{错误处理和调试信息}
\label{app:error_handling}

\subsection{常见错误及解决方案}
\label{app:common_errors}

\begin{table}[h]
\centering
\caption{实验过程常见错误及解决方案}
\label{tab:error_solutions}
\begin{tabular}{p{4cm}p{6cm}p{4cm}}
\toprule
\textbf{错误类型} & \textbf{错误描述} & \textbf{解决方案} \\
\midrule
CUDA内存不足 & GPU内存溢出 & 减小batch\_size或使用梯度累积 \\
数据加载超时 & DataLoader工作进程卡死 & 设置num\_workers=0或增加timeout \\
模型不收敛 & 损失震荡或发散 & 降低学习率或增加正则化 \\
CSI数据格式错误 & 复数矩阵维度不匹配 & 检查数据预处理管线 \\
跨域评估失败 & LOSO/LORO划分错误 & 验证元数据和划分脚本 \\
\bottomrule
\end{tabular}
\end{table}

\subsection{调试和验证脚本}
\label{app:debug_scripts}

\begin{lstlisting}[language=bash, caption=实验验证脚本]
#!/bin/bash
# 验证实验环境和数据完整性

echo "=== 环境检查 ==="
python -c "import torch; print(f'PyTorch: {torch.__version__}')"
python -c "import torch; print(f'CUDA available: {torch.cuda.is_available()}')"

echo "=== 数据完整性检查 ==="
python scripts/validate_data_integrity.py --dataset synthetic
python scripts/validate_data_integrity.py --dataset real

echo "=== 模型结构验证 ==="
python scripts/validate_model_architecture.py --model Enhanced
python scripts/validate_model_architecture.py --model CNN

echo "=== 实验配置验证 ==="
python scripts/validate_experiment_configs.py --protocol D2
python scripts/validate_experiment_configs.py --protocol CDAE
python scripts/validate_experiment_configs.py --protocol STEA

echo "✅ 所有验证通过,可以开始实验"
\end{lstlisting}

\section{附录小结}
\label{app:details_summary}

本附录提供了博士论文实验研究的所有技术细节,确保:
\begin{itemize}
\item \textbf{完全可重现性}:硬件、软件、参数配置全记录
\item \textbf{技术透明性}:算法实现、评估方法完全公开
\item \textbf{统计严谨性}:显著性检验、置信区间标准化
\item \textbf{工程可用性}:错误处理、调试工具完备
\end{itemize}

这些详细信息支撑了主文实验章节的科学性和工程实用性,为WiFi CSI人体活动识别技术的进一步研究和应用提供了坚实的技术基础。